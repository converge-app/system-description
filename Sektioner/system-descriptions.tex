\chapter{Overordnet om systemet}

Vi lever i en moderne tid, hvor der ekstremt meget fokus på personlig frihed og selvstændighed. Det er blevet i større grad populær at folk ønsker og have frihed under eget ansvar, det vil sige at komme ud af den klassiske arbejdsstruktur. Til det er der blevet udviklet en salgsplatform (Converge), hvor hovedformålet er at modernisere og effektivisere arbejdskraft, i form af udbyde og sælge arbejdskraft.  På Converge platformen vil udbydere have mulighed for at få løst kort eller langvarigt projekt, efter deres ønske. Fokusområdet vil være interaktionen mellem udbydere og arbejdstager (freelancer).  Dette betyder at udbydere og freelancer skal have, så meget personlige kontakt så muligt, for at kunne opbygge noget tillid/sikkerhed. 

\begin{figure}[H]
    \begin{small}
        \begin{center}
            \includegraphics[width=0.65\textwidth]{software-architecture/domain-model/domain-model.pdf}
        \end{center}
        \caption{System domæne}
        \label{fig:system-domain}
    \end{small}
\end{figure}

For at tilgå Converge platformen, skal hver bruger oprette sig, herefter vil brugeren have alle platformens funktionalitet til rådighed. Når brugeren har oprettet sig og er logget ind i systemet, vil der være mulighed for at oprette en personlig portfolio, alt efter om man er udbyder eller freelancer. Portfolio vil indeholde nogle generelle informationer om brugeren (navn, stilling m.m.) og nogle specifikke oplysninger om brugerens kompetencer (C, 3D designer, C\# m.m.).  Derudover er det muligt for brugeren at oprette et projekt, som der ønskes og få løst. Når projektet er oprettet, har freelancer mulighed for at finde et projekt efter eget præferencer og byde på det, og så er det udbydere der vælger en freelancer efter eget ønske til at få løst projektet. Udbydere kan også se de oprettet projekter, som han/hun har oprettet.  Converge platform giver brugeren mulighed for at benytte platformens integreret betalingssystem til at betale eller hæve penge ud. Her kan brugeren se hvor mange penge han har på sin konto, status over de forskellige handlinger brugeren har fortaget og hvornår der er blevet overført/modtaget penge. Derudover kan brugeren uploade filer til andre bruger, så der kan ske en udveksling af materiale mellem udbydere og freelancer. Som sikkerhed indeholder platformen en fuldt audit-trail, så både freelancer og udbyder er i stand til at få et overblik over alle aktioner fra de forskellige parter. Samtidig tilbyder Converge et chatsystem, så brugerne kan chatte med hinanden, da det giver et mere personligt kendskab mellem bruger og på denne måde opbygge et tillidsbånd.   

\begin{figure}[H]
    \begin{small}
        \begin{center}
            \includegraphics[width=0.95\textwidth]{components/architecture-backend-components/Components.pdf}
        \end{center}
        \caption{System komponenter}
        \label{fig:system-components}
    \end{small}
\end{figure}
